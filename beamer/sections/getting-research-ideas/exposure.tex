\begin{frame}{Exposure Yourself to Ideas}
  Set aside some time every week for trying to generate research ideas. Some
  possible catalysts are:

  \begin{itemize}
    \item Make a weekly trip to the library to read at least the abstracts from the
      premier journals in your eld. Choose an article or two to read in depth
      and critique.

    \item Make a weekly investigation to nd technical reports in your eld, using
      electronic resources or libraries. Read selectively and critique.
    
    \item Attend at a research seminar or colloquium series. Listen and critique.
  \end{itemize}

  Add these to your log, and ask the canonical questions. As you review the
  log 6 months from now, you may nd something that strikes a chord then but
  is beyond you now.\\

  \textit{It is very important to make the transition from the passive mode of learning.}
\end{frame}

\begin{frame}{Directed Study}
  Which comes first: the thesis advisor or the thesis topic? The answer is, both
  ways work.\\\vspace{2em}

  If you have identified a compatible advisor, you could ask for an
  independent study course. Both of you together set the focus for the course,
  with you having more or less input depending upon your progress in identifying
  a subfield of research.
\end{frame}

\begin{frame}{Developing the Germ of an Idea}
  Once you have identified a topic that looks feasible, make sure you are aware of
  all of the literature in the area. Keep reading and listening, and keep distinct
  in your mind what is different between your work and others. If you do not
  frequently review the literature you read months ago, you may find yourself
  unconsciously claiming credit for other people's ideas.\\\vspace{2em}%

  On the other hand, don't let other people's frame of mind limit your creativity.
\end{frame}
