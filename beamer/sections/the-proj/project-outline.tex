\begin{frame}{The Project}
    \begin{center}
        Research in computer science often leads to a “project” involving programming.\vspace{2em}

        It's important to remember that programming is not computer science research. Instead, for
        most computer scientists, programming is merely a mechanism for performing an experiement.\vspace{2.5em}

        As with any experiment, it should be carefully planned, ahead of time\ldots
    \end{center}
\end{frame}


\begin{frame}{The Project}
    \begin{itemize}
        \only<1>{\item{\textbf{Establish goals.} Know where you are headed, and approach the solution without
             distraction. Develop a list of milestones which demonstrate progress, and strive to
             accomplish them. If you cannot formulate concise goals, you should stop and reconsider the
             motivations for the project.\vspace{2em}

             Goals should be `justified' regularly. As your knowledge and understanding improve, so may
             your ideas and motivations. Dynamic goals provide flexibility as you undergo research; \textit{don't 
             be afraid to scrap a goal in favour of another.}
        }
        }
    
        \only<2>{\item{\textbf{Think simple.} Design your projects so that they may be completed within a reasonable
            period of time. An experienced programmer generates little more than a hundred lines of
            reliable code per month. A project that demands thousands of lines of code, therefore, will
            take more than a couple of two-week blocks to implement correctly. Time spent pruning the
            experiment to a manageable size is time well spent. Large projects do not necessarily yield
            large results.
    }}
          
        \only<3>{\item{\textbf{Build prototypes.} Most projects benefit from the construction of a prototype. A well
            considered prototype validates assumptions, tests the value of abstractions, and motivates
            reconsideration of weak ideas. While there is little research value associated with polishing
            off a `product', many research questions can be answered satisfactorily through mock-ups
            or partial implementations.
    }}
          
    \only<4>{\item{\textbf{Use tools.} A programmer’s performance is dramatically improved through the use of a
            few simple tools. There are, of course, many important and useful tools, but the main point
            is clear: \textit{the correct choice of tools can reduce the total work in a project.} Find them,
            learn from them, and use them.
    }}
          
        \only<5>{\item{\textbf{Collaborate.} When resources can be coordinated, groups are often more productive than
            individual efforts in isolation. For many, success comes from collaboration. Try to share
            and develop ideas in a group atmosphere. Contact others that share your interestes and
            collaborate! Undoubtedly they will have solved problems you are currently considering, and their
            solutions will influence how you achieve common goals.\vspace{1em}
          
            \textit{One side effect of collaboration is increased discipline, discipline that is necessary to reduce
            the amount of energy expended synchronizing efforts.} Some tips for collaboration:
            \begin{itemize}
                \item{Keep a regular schedule of meetings}
                \item{Establish points of synchronization where concentration is refocused on an issue}
                \item{Consider critisim carefully; conversely, provide others only with constructive critisim}
            \end{itemize}
    }}
       
        \only<6>{\item{\textbf{Document results.} Finished projects should be documented. At a minimum, a technical 
           overview of the experiment allows others to see what motivates your research. The
           document should describe the problem, your assumptions, your approach, and an honest
           evaluation of your results. When documenting software, include illustrative examples,
           tutorials, and any experience gained from its use. Well written documentation greatly
           increases the impact of a project.\vspace{2em}
         
           \textit{It's vital to reserve a good portion of time for writing.} An hour spent writing is an hour spent
           considering a problem instead of, say, grappling with a computer. You must spend time away from
           other distractions to document work and focus your efforts.
   }}
  \end{itemize}
\end{frame}
