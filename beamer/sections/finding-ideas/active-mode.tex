\begin{frame}{Active Readering and Listening}
    \begin{center}
        \Large{It's very important to make the transition \textbf{from} the \textbf{passive} mode of learning
        that traditional lecture courses encourage \textbf{to} an \textbf{active and critical} learning style.}
    \end{center}
\end{frame}

\begin{frame}{Active Readering and Listening}
  Whenever you read technical material, evaluate a piece of software, or
  listen to a research talk, ask yourself these canonical questions:
  \begin{itemize}
    \uncover<2->{
      \item{From where did the author seem to draw the ideas?}
    } 
    \uncover<3->{
      \item{What exactly was accomplished by this piece of work?}
    } 
    \uncover<4->{
      \item{How does it seem to relate to other work in the field?}
    }
    \uncover<5->{
      \item{What would be the reasonable next step to build upon this work?}
    }
    \uncover<6->{
      \item{What questions are left unanswered?}
    }
    \uncover<7->{
      \item{What are the important references cited by this work?}
    }
    \uncover<8->{
      \item{What ideas from related elds might be brought to bear upon this subject?}
    }
    \uncover<9->{
      \item{Can the results be generalized?}
    }
  \end{itemize}

  \uncover<10->{
    \begin{center}
        Try to keep a written log of your technical reading and listening. 
    
        Review it periodically to see if some of the ideas begin to fit together. 
    \end{center}
  }
\end{frame}

