\begin{frame}{Exposing Yourself to Ideas}
    Set aside some time every week/month for trying to generate research ideas or improving on exisiting ideas.

    \begin{itemize}
        \uncover<2->{
          \item{Finding and reading related work is the foundation of good research.}
        }
        \uncover<3->{
            \item{If you're new to the field, you may only be familiar with buzzwords and/or material presented in
            textbooks. Some important initial resources for finding pertinent references are:}
            \begin{itemize}
                \item{\footnotesize{\href{https://libraries.acm.org/digital-library}{The Association for Computing Machinery (ACM) Digital Library (DL)} }}
                \item{\footnotesize{\href{https://cra.org/resources/}{The Computing Research Association (CRA)}}}
                \item{\footnotesize{\href{https://arxiv.org/corr}{The arXiv Computing Research Repository (CoRR)}}}
                \item{\footnotesize{Github/Gitlab, \href{http://phrack.org/}{Phrack}, \href{https://pagedout.institute/}{PagedOut!}}}
                \item{\footnotesize{\href{https://media.defcon.org/}{DEF CON Media Server}}}
                \item{\footnotesize{\href{https://media.ccc.de/}{CCC Media Server}}}
            \end{itemize}
        }
        \uncover<4->{
            \item{If there's a mailing list, subscribe to it. Trying to pwn something as part of the
            research? \textit{Subscribe to the developer mailing list.}}
        }
    \end{itemize}

    \uncover<5->{\centering{\textit{\textbf{How can these ideas help me solve my research problem?}}}}
\end{frame}

\begin{frame}{Exposing Yourself to Ideas}
    Struggling with remaining active? Try one (or more) of these catalysts:
    \begin{itemize}
        \item{ Make a monthly investigation to read, at least, the recent abstracts from
            a given journal. Choose an article or two to read in-depth and critique.}
        \item{Make a weekly investigation to find technical talks/videos in your field. View selectively and critique.}
        \item{Pick a random conference (DEFCON/BH/CCC/etc.) talk or series. Listen and critique.}
        \item{Search common code repository sites for keywords relevant to the research. Learn from and/or contribute
            to any that you find. Critique it's approach.}
    \end{itemize}
    
    \begin{center}
    Add these to your log, and ask the canonical questions.
    \end{center}
\end{frame}


\begin{frame}{Becoming Part of the Community}
    %One of the most imporants things a researcher should do, for both themselves and their
    %work, is establish themselves as part of the research community:
    \footnotesize{
        \textbf{Join a community:} Local IRL or online (slack/discord/zulip/etc) communities provide
        an informal and casual settings to learn, teach, and bounce ideas.
        \textit{This is especially import for researchers new to the field.}\vspace{1em}

        \textbf{Attend conferences/workshops:} Conferences and workshops are the best place to meet people,
        discuss your ideas, hear new ideas, and get a sense for the current state of research in a field.\vspace{1em}
        
        \textbf{Publish papers:} Publishing papers provides a source of feedback and forces
        you to clarify the ideas presented and to fit them into the context of the current state of research.\vspace{1em}

        \textbf{Deliver talks:} Talks are a great way gain visibility, share your ideas, and hear 
        them out-loud. Submit to CFPs at the next conference; lesser-known conferences
        provide the ideal `low-pressure' environment for a new speaker!\vspace{1em}

        \textbf{Networking:} Talk about your research interests [to willing participants] every chance you get (but make sure
        to spend time \textit{listening}, too -- this is when you'll learn the most)
    }
\end{frame}
